\documentclass[a4paper,11pt]{jreport}

% slis修論のスタイルシート
\usepackage{Master_ja/masterThesisJa-Ja}

% 画像読み込み用の定番パッケージ
\usepackage[dvipdfmx]{graphicx}

% 画像が多くなってタイプセットに時間がかかるときはこっちに切り替える
% \usepackage[dvipdfmx, draft]{graphicx}

% texファイルの分割のためのパッケージ
\usepackage{subfiles}

% 本文中のURLにハイパーリンクを設定するパッケージ
\usepackage{url}

% 題目提出したら反映させましょう
\maintitle{日本語タイトル}{English Title}
\publish{2020}{03}
\student{201821000}{筑波 太郎}{Taro TSUKUBA}

% 要旨
% !TEX root=../main.tex
\abst{ 
  筑波大学では、教育・研究・運営の全般にわたって一層の機能強化を図るために教育研究体制の改革を実施し、
  本学独自の新たな教員組織として「系」が設置されました。
  筑波大学では全体で10の系があり、すべての教員は原則としていずれかの「系」に所属します。
  図書館情報メディア系では、図書館情報メディアという対象を様々な側面から研究しています。 
  人類の科学技術が作り上げた様々な情報メディアは、知的創造基盤として我々の生活に欠くことのできないものになっておりますが、
  それと同時に我々の生活のあり方を大きく変えるものでもあります。
  その進歩の歴史と現状を明らかにし、その方向性を定めてそれを実現していくことは、
  我々人類の持続的な発展のために必須の課題です。
  そのためには人文・社会・科学技術などの多様な手法を総合的・学際的に用いた研究を実現することが必要です。
  そして本系の教員は、図書館情報メディア研究の推進に多くの寄与をしつつ、その課題の解明に向かって着実に取り組んでおります。\cite{what_is_slis}
}

% 指導教員の名前
\advisors{筑波 花子}{筑波 次郎}

\begin{document}

% masterThesisJa.styで表紙・英表紙・抄録を作るコマンド
\makecover

\addtolength{\textheight}{-5mm}	% 本文の下限を5mm上昇
\setlength{\footskip}{15mm}	% フッタの高さを15mmに設定
\fontsize{11pt}{15pt}\selectfont

\newpage
\pagestyle{plain}
\raggedright

% ページ番号を小文字のローマ数字にする
\pagenumbering{roman} % I, II, III, IV 

% 目次を表示
\tableofcontents
\newpage

% 図目次を表示
\listoffigures
\newpage

% 表目次を表示
\listoftables
\newpage

% 本文
\parindent=1zw	% インデントを1文字分に設定
\pagebreak\setcounter{page}{1}
\pagenumbering{arabic} % 1,2,3
\pagestyle{plain}

% はじめに
\subfile{chapter/01_intro}

% 手法
\subfile{chapter/02_method}

% 結果
\subfile{chapter/03_result}

% 考察
\subfile{chapter/04_discussion}

% まとめ
\subfile{chapter/05_conclusion}

% 謝辞
\chapter*{謝辞}
% !TEX root=../main.tex
この論文を執筆するに当たり,手厚くご指導してくださった本学図書館情報メディア系の****先生に深く感謝いたします.
また,同研究室の学生の皆様にも感謝の意を評します.
最後に,私の人生を支えてくださった両親に心より感謝します.

% 目次に謝辞を表示させる
\addcontentsline{toc}{chapter}{\numberline{}謝辞}

\renewcommand{\bibname}{参考文献}
\bibliographystyle{junsrt}
\bibliography{ref}

% 目次に参考文献を表示させる
\addcontentsline{toc}{chapter}{\numberline{}参考文献}

\end{document}
