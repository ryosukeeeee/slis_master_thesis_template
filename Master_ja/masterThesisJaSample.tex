\documentclass[a4paper]{jreport}

\usepackage{masterThesisJa}
\setcounter{tocdepth}{3}
\setcounter{page}{-1}

% 【必須】主題:\maintatile{日本語}{英語}
\maintitle{国文学論文の情報検索}{Information Retrieval for Articles on Japanese Literature}

% 【任意】副題:\subtitle{日本語}{英語}
% 副題が不要な場合は次の行をコメントアウトしてください
\subtitle{—図書館情報メディア研究科の場合—}{: A Case Study at the Graduate School of Library, Information, and Media Studies}

% 【必須】発表年月:\publish{年}{月}
\publish{20XX}{XX}

% 【必須】学生情報:\student{学籍番号}{氏名(日本語:氏名の間は1文字空ける)}{氏名(英語:名は先頭のみ大文字、姓は全て大文字)}
\student{20XX21XXX}{筑波 太郎}{Taro TSUKUBA}

% 【必須】概要:\abst{概要}
\abst{
 この文書は,筑波大学大学院図書館情報メディア研究科の修士論文概要のサンプルである.*************************************************************************\\ 
 *********************************************************************************************************************\\
 ***********************************************************************************************************\\
 このサンプルは同じような書式にするために提供している.
}

% 【必須】研究指導教員(氏名の間は1文字空ける):\advisors{主研究指導教員}{副研究指導教員}
\advisors{大学 一郎}{紫峰 花子}


% 以下、本文を出力
\begin{document}

\makecover

\addtolength{\textheight}{-5mm}	% 本文の下限を5mm上昇
\setlength{\footskip}{15mm}	% フッタの高さを15mmに設定
\fontsize{11pt}{15pt}\selectfont

% 目次・表目次を出力
\newpage
\raggedright
\pagenumbering{roman} % I, II, III, IV 
\tableofcontents
\listoffigures

% 本文
\parindent=1zw	% インデントを1文字分に設定
\pagebreak\setcounter{page}{1}
\pagenumbering{arabic} % 1,2,3
\pagestyle{plain}

% 章:\chapter{}
% 節:\section{}
% 項:\subsection{}

\chapter{序章}

\chapter{関連研究}
\section{○○の研究}
\subsection{○○における××の研究}

\chapter{提案}

\chapter{評価実験}

\chapter{考察}

\chapter{まとめ}

\chapter*{謝辞}


% 参考文献(References)
\newpage
\addcontentsline{toc}{chapter}{\numberline{}参考文献}
\renewcommand{\bibname}{参考文献}

%% 参考文献に bibtex を使う場合
%\bibliographystyle{junsrt}
%\bibliography{hoge}

%% 参考文献を直接ファイルに含めて書く場合
\begin{thebibliography}{99}

% e.g.)
% \bibitem{studyA}
% 筑波太郎,土浦花子,「○○○○・・・に関する研究」,日本○○○○学会誌,Vol.2,No.3,2016,pp.234--240.

\end{thebibliography}

\end{document}