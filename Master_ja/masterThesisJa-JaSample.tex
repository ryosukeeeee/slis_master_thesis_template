\documentclass[a4paper]{jreport}	% 日本語の場合

\usepackage{masterThesisJa-Ja}
\setcounter{tocdepth}{3}
\setcounter{page}{-1}

% 【必須】主題:\maintatile{日本語}{英語}
\maintitle{国文学論文の情報検索}{Information Retrieval for Articles on Japanese Literature}

% 【任意】副題:\subtitle{日本語}{英語}
% 副題が不要な場合は次の行をコメントアウトしてください
\subtitle{—図書館情報メディア研究科の場合—}{: A Case Study at the Graduate School of Library, Information, and Media Studies}

% 【必須】発表年月:\publish{年}{月}
\publish{20XX}{XX}

% 【必須】学生情報:\student{学籍番号}{氏名(日本語:氏名の間は1文字空ける)}{氏名(英語:Twins登録の表記)}
\student{20XX21XXX}{筑波 太郎}{Tsukuba Taro}

% 【必須】概要:\abst{概要}
\abst{
 本研究では,インタフェースとしてタッチパネルを使用した携帯電話を開発した.従来テンキーと十字キー,および数個のボタンで操作していた携帯電話だが,本研究で開発した携帯電話ではボタンは3つのみ搭載し,ほぼ全ての操作をタッチパネルを通して行う.我々はこの携帯電話のインタフェースについて,利便性が従来の携帯電話に比べ向上したことを報告する.
}

% 【必須】研究指導教員(氏名の間は1文字空ける):\advisors{主研究指導教員}{副研究指導教員}
\advisors{大学 一郎}{紫峰 花子}


% 以下,本文を出力
\begin{document}

\makecover

\addtolength{\textheight}{-5mm}	% 本文の下限を5mm上昇
\setlength{\footskip}{15mm}	% フッタの高さを15mmに設定
\fontsize{11pt}{15pt}\selectfont

% 目次・表目次を出力
\pagebreak\setcounter{page}{1}
\pagenumbering{roman} % I, II, III, IV 
\tableofcontents
\listoffigures

% 本文
\parindent=1zw	% インデントを1文字分に設定
\pagebreak\setcounter{page}{1}
\pagenumbering{arabic} % 1,2,3
\pagestyle{plain}

% 章:\chapter{}
% 節:\section{}
% 項:\subsection{}

\chapter{序章}
% プレースホルダ
現在,遠隔地間で通話する方法として携帯電話が使用されている.ユーザは携帯電話を使ってどこででも通話することができ,これまで固定電話や公衆電話を利用して行われていた仕事相手や友人同士の連絡がより容易になった.

\chapter{関連研究}
\section{○○の研究}
\subsection{○○における××の研究}

\chapter{提案}

\chapter{評価実験}

\chapter{考察}

\chapter{まとめ}

\chapter*{謝辞}


% 参考文献(References)
\newpage
\addcontentsline{toc}{chapter}{\numberline{}参考文献}
\renewcommand{\bibname}{参考文献}

%% 参考文献に bibtex を使う場合
%\bibliographystyle{junsrt}
%\bibliography{hoge}

%% 参考文献を直接ファイルに含めて書く場合
\begin{thebibliography}{99}

% e.g.)
\bibitem{studyA}
黒田治之,千葉和彦,「M.26 わい性台木利用リンゴ樹における生産構造と光環境に及ぼす栽植密度の影響」,園芸学会雑誌,Vol.71,No.4,2002,pp.544-552.
\end{thebibliography}

\end{document}